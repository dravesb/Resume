%%%%%%%%%%%%%%%%%%%%%%%%%%%%%%%%%%%%%%%%%
% Medium Length Professional CV
% LaTeX Template
% Version 2.0 (8/5/13)
%
% This template has been downloaded from:
% http://www.LaTeXTemplates.com
%
% Original author:
% Trey Hunner (http://www.treyhunner.com/)
%
% Important note:
% This template requires the resume.cls file to be in the same directory as the
% .tex file. The resume.cls file provides the resume style used for structuring the
% document.
%
%%%%%%%%%%%%%%%%%%%%%%%%%%%%%%%%%%%%%%%%%

%----------------------------------------------------------------------------------------
%	PACKAGES AND OTHER DOCUMENT CONFIGURATIONS
%----------------------------------------------------------------------------------------

\documentclass{resume} % Use the custom resume.cls style
\usepackage{hyperref}
\usepackage{multicol}
\usepackage[usenames,dvipsnames]{xcolor}

\usepackage[left=0.75in,top=0.6in,right=0.75in,bottom=0.6in]{geometry} % Document margins

\name{Benjamin Draves} % Your name

\address{615 West Monroe Street, Easton, PA 18042} % Your address
\address{\tt{
\hfill (330) 428-5025\hfill \href{mailto:dravesb@lafayette.edu}{dravesb@lafayette.edu}
 \hfill \href{https://github.com/dravesb}{https://github.com/dravesb} \hfill}}

\begin{document}

\begin{rSection}{Education}
{\bf Lafayette College} (August 2014 - Present) \hfill {\em Overall GPA: 3.86 (In Major: 3.80)} \\ 
B.S. in Mathematics, Minor in Computer Science \hfill Degree expected May, 2017

{\bf Alliance High School} \hfill {\em GPA: 4.0} \\ 
Graduated Valedictorian \hfill August 2010 - May 2014
\end{rSection}
%------------------------------------------------

\begin{rSection}{Experience}

\begin{rSubsection}{Consulting}{September 2015- Present}{Consultant}{Easton, PA} 
\item Crayola.com- Applied mixed effects models to Crayola sales data to optimize online sales and advise marketing strategies.
\item Easton Area Neighborhood Center (EANC)- Designed, piloted, and distributed a public opinion survey to inform future programming of EANC. 
\item Easton City Schools (ECS)- Analyzed the effects of the Classroom Diagnostic Tool (CDT) on student's Keystone exams to inform circular design of ECS. 
\end{rSubsection}

\begin{rSubsection}{Research}{June 2015 - Present}{Undergraduate Research}{Easton, PA}
\item Treelet Covariance Smoothing with an application in genetic clustering (\textit{ongoing}). Research mentor: Dr. Trent Gaugler. Funded by EXCEL scholarship committee, Lafayette College. 
\item Tempo of the Times - Conducted original research examining the connection between musical compositions and the social climate. Research was presented at Bucknell Digital Scholarship Conference. Hosted at \href{http://tempoofthetimes.com}{tempoofthetimes.com}.  

\end{rSubsection}

\begin{rSubsection}{Github Projects}{April 2015-Present}{}{}
\item My other current projects can be found on my blog \href{http://dravesb.github.io}{dravesb.github.io}.
\end{rSubsection}

\end{rSection}
%------------------------------------------------
\begin{rSection}{Relevant Coursework}
\textbf{Mathematics:} Transition to Theoretical Mathematics, Differential Equations \& Linear Algebra, Vector Spaces, Abstract Algebra, Real Analysis, Complex Analysis



\item \textbf{Statistics:} Probability, Statistics, Mixed Effect Modeling, Time Series Analysis, Survey Design \& Analysis, Operations Research


\item \textbf{Computer Science:} Introduction to Computer Science, Data Structures and Algorithms, Agent Based Modeling

\end{rSection}
%------------------------------------------------
\begin{rSection}{Programming Capabilities}
\textbf{Proficient:} Java, R, SAS\\
\textbf{Intermediate:} Python\\
\textbf{Typesetting:} Latex, Microsoft Office

\end{rSection}

%------------------------------------------------


\begin{rSection}{Activites \& Honors}
Pi Mu Epsilon (Mathematical Honor Soceity)\hfill March 2015 - Present\\
Founding Member of the Statistics House\hfill August 2015 - Present \\
Treasurer of the Mathematics Club\hfill August 2015 - Present \\
Treasurer of College Democrats\hfill August 2015 - Present
\end{rSection}

\end{document}
